\begin{chapter}{Ablaufplanung}
  \f{Unterscheidungskriterien}
  \begin{itemize}
   \item statische vs. dynamische -- statische Ablaufpläne treten auf, wenn zur Übersetzungs-/Synthesezeit ein Ablaufplan erstellt werden muss, vor allem bei hardwarenahem Implementationstechnicken. Dynamische Ablaufpläne treten zur Laufzeit des ES auf, meist bei Softwareimplementierungen.
   \item präemptiv vs nicht-präemptiv -- bei präemptiven Plänen geht man davon aus, dass die Laufzeit eines Jobs viel länger ist als die Dauer, ihn zu pausieren, meist Software. Bei kurzen oder nicht-unterbrechbaren Jobs verwendet man nicht-präemptive Pläne.
   \item Abhängigkeitsbedingungen -- bei Datenabhängigkeiten unterliegt der Plan zeitlichen Einschränkungen, gegeben durch DAG. Kommen sehr häufig vor.
   \item Ressourcenbeschränkungen -- Hardware (PUs, Busse, Speicher) als Nebenbedingung, Probleme werden schwerer.
   \item Zeitbeschränkungen -- 
   \begin{itemize}
    \item absolute: Deadlines \& Release-Termine
    \item relative: zeitliche Bedingungen zwischen den einzelnen Jobs (Start- \& Endzeitpunkte)
   \end{itemize}
   \item periodisch vs. aperiodisch -- periodische Probleme kommen bei Iterationen vor, periodische Einplanung. Findet Verwendung bei Schleifenoptimierungen oder DSP.
  \end{itemize}
  
  \begin{section}{Statische Ablaufplanung}
   \begin{subsection}{Statische Ablaufplanung ohne Ressourcenbeschränkungen} 
    ist einfach und kann von gängigen Graph-Algorithmen effizient gelöst werden.
    
   \f{der As Soon as Possible-Algorithmus (ASAP) (Komplexität $O(|V| + |E|)$)}
   \begin{itemize}
    \item berechne topologische Sortierung
    \item verplane jeden Knoten in Reihenfolge der topologischen Sortierung vom frühest möglichen Zeitpunkt (wenn die entsprechende Ressoure wieder frei ist)
   \end{itemize}

   Dieser Algorithmus liefert einen Ablaufplan minimaler Latenz $\rightarrow$ untere Schranke für Pläne mit Zeit- \& Ressourcenbeschränkungen. (Beweisbar über Induktion über die Startzeitpunkte.) 
   
   \f{der As Late as Possible-Algorithmus (ALAP) (Komplexität $O(|V| + |E|)$)}
   \begin{itemize}
    \item berechne umgekehrte topologische Sortierung
    \item verplane jeden Knoten in Reihenfolge der umgekehrten Sortierung vom spätest möglichen Zeitpunkt (beachte kritische Pfade)
   \end{itemize}

   Dieser Algorithmus findet Anwendung, wenn eine Latenzschranke vorgegeben ist und die Jobs möglichst spät gestartet werden sollen.
   \end{subsection}
   
   \begin{subsection}{Ablaufplanung mit Zeitbeschränkungen}
    absolute Zeitbeschränkungen $\Leftrightarrow$ relative Zeitbeschränkungen zum Startknoten $\Rightarrow$ wir betrachten nur relative Zeitbeschränkungen.\\
    
    Der \f{Constraintgraph} ist ein gewichteter, gerichteter Graph mit Gewichtsfunktion, die sich aus dem Sequenzgraphen ergibt. Er enthält alle Knoten \& Kanten des Sequenzgraphen sowie für jede Zeitbeschränkung eine weitere Kante, deren Gewicht der Zeitbeschränkung entspricht.\\
    
    Die Existenz eines gültigen Ablaufplans kann mit Zeit- aber ohne Ressourcenbeschränkungen in polinomieller Zeit entschieden und ein latenzminimaler, gültiger Ablaufplan $\uptau$ -- falls existent -- gefunden werden. (Komplexität $O(|V_S|\cdot|E_C|)$ mittels Bellman-Ford)\\
   \end{subsection}
   
   \begin{subsection}{Ablaufplanung mit Ressourcenbeschränkungen}
    \f{Optimierungsprobleme}
    \begin{itemize}
     \item Latenzminimierung mit Ressourcenbeschränkungen -- bei gegebener Allokation sind Bindung und Ablaufplan mit minimaler Latenz gesucht
     \item Kostenminimierung unter Latenzbeschränkung -- bei gegebener Latenzschranke sind Allokation, Bindung und Ablaufplan mit minimalen Kosten gesucht
     \item Zulässiges Ablaufproblem -- gesucht ist eine Implementierung bei gegebener Latenzschranke und Allokation
     \item Gewichtete Minimierung von Latenz und Kosten -- gesucht ist eine Implementierung, die bei gegebener Latenzschranke und Allokation eine Kostenfunktion minimiert
    \end{itemize}
    
    \f{List Scheduling} ist eine Weiterentwicklung von ASAP. Es werden Listen geführt für alle abgearbeiten Jobs, alle offenen Jobs und alle Ressourcen mit ihren darauf laufenden Jobs. Wähle eine maximale Teilmenge aller Jobs je Ressourcentyp mit absteigender Priorität.  Plane diese Jobs sinnvoll ein. Die Qualität des Scheduling hängt von der Wahl der Knoten ab. Gebräuchliche Prioritätsfunktionen:
    \begin{itemize}
     \item Anzahl der Nachfolgeaufträge je Job
     \item Länge des längsten Pfades vom Job zum Ende
     \item Mobilität der Aufgaben (Differenz zwischen ASAP und ALAP Startzeiten)
    \end{itemize}
    List Scheduling ist nur eine Heuristik!\\
    
    \f{Force Directed Scheduling} benutzt ein Kräftemodell, um die Jobs an geeignete Ausführungszeitpunkte zu schieben/ziehen. Dynamische Änderung des Modells nach Planung von Aufgaben, Berücksichtigung direkter Nachbarschaften im Graphen, Updates des Kräftemodells nach jedem Schritt. Man beachtet dabei folgende Kräfte:
    \begin{itemize}
     \item Selbstkraft -- Ausführung am besten zu Zeiten geringer Auslastung
     \item Nachbarschaftskräfte -- Einfluss durch Vorgänger- \& Nachfolger-Jobs. Planung nach sinkender mittlerer Auslastung der Ressourcen
    \end{itemize}
    Benutze diese Kräfte als Priorisierung für das List Scheduling
    \begin{itemize}
      \item wähle die höchsten Kräfte für minimale Latenz unter Ressourcenbeschränkungen
      \item benutze Kräfte, um Job für Job Zeitpunkten zuzuordnen für minimalen Ressourcenbedarf unter Latenzschranke (heuristische Minimierung der Auslastung)
    \end{itemize}
   \end{subsection}
   
   \begin{subsection}{Ablaufpläne mittels ILP}
    ILPs sind ganzzahlige lineare Programme. Sollte ein Ablaufplan mit Bindung existieren, so wird dieser auch vom ILP gefunden. (Lösen mittels Branch\&Bound, Relaxierung findet untere Schranken)
   \end{subsection}
  \end{section}
  
  \begin{section}{Dynamische Ablaufplanung}
   Es liegt eine iterative Definiton der Berechnungen vor, also immer die gleichen Jobs abhängig vom Zeitindex $n$.
   
   Ein \f{iterativer Problemgraph} ist ein Netzwerk, dessen Kantengewichte die Indexverschiebungen darstellen -- dieser kann insbesondere Zyklen beinhalten. Ein periodischer Ablaufplan ordnet jedem Job eine Menge von Startzeitpunkten zu, die die Indexverschiebungen berücksichtigen. Das Iterationsintervall ist die Menge aller Zeitschritte zwischen Start des ersten Jobs und Ende des letzten Jobs einer Iteration. Die Art der \f{Parallelisierung} hat starken Einfluss auf die Datenrate, mit der die Jobs verarbeitet werden können.
   \begin{itemize}
    \item sequenzielle Abarbeitung -- erforderlich, wenn alle Daten einer Iteration mit einem Takt synchronisiert werden müssen
    \item nichtüberlappende Abarbeitung -- erforderlich, wenn sich alle Jobs nach einer Iteration synchronisieren müssen. Äquivalent zur sequenziellen Abarbeitung bei geeignetem Retiming (Indexverschiebung)
    \item überlappende Abarbeitung -- ermöglicht viel kürzere Perioden bei geeigneter Planung
   \end{itemize}
   
   \begin{itemize}
    \item vollstatische Bindung -- jeder Job läuft in jeder Iteration stets auf der gleichen Ressourceninstanz
    \item zyklostatische Bindung -- jeder Job läuft nach jeder $k$-ten Iteration wieder auf der gleichen Ressourceninstanz
   \end{itemize}
   
   \begin{subsection}{Sequenzielle periodische Ablaufplanung}
    \f{Satz:} Es gibt einen gültigen, sequentiellen, periodischen Ablaufplan ohne Ressourcenbeschränkung mit Iterationsintervalllänge $L$ für einen Problemgraphen $G=(V,E,s)$, genau dann, wenn
    \[ \forall u,v\in V: W(u,v)>L \Rightarrow S(u,v) \geq 1 \]
    Dabei ist $S(u,v)$ die Menge der Längen der kürzesten Pfade von $u$ nach $v$.
   \end{subsection}
   
   \begin{subsection}{Retiming}
    Wir versuchen durch Indexverschiebungen die Länge des Iterationsintervalls zu minimieren $\rightarrow$ neufestlegung der Interationsindizes. Ein Retiming $r(u)$ ändert im Problemgraphen nur die Größen $S(u,v)$ -- und auch nur in Abhängigkeit von $r$ -- jedoch nicht die Größe $W(u,v)$. Die Lösung eines gültigen Retimings ist also eigentlich ein single-source-longest-path Problem in einem besonderen Graphen $G_{r,L}$. Gibt es in diesem Graphen einen positiven Zyklus, so gibt es kein Retiming der Länge $L$.
    
    \f{Retiming-Algorithmus}
    \begin{itemize}
     \item Berechne zu einem Problemgraphen die Größen $S(.)$ und $W(.)$ durch Lösung eines all pair shortest path Problems $O(|V||E|log|V|)$
     \item Sortiere die Menge $\{ W(u,v) | u,v \in V \}$ $O(|V|2log|V|)$
     \item Bestimme durch Binärsuche das kleinste $L=W(u,v)$ aus obiger Menge, für das single source longest path Problem in $G_{r,L}$ eine Lösung $r$ hat. Wähle $r$ als Retiming. $O(|V|3log|V|)$
    \end{itemize}
   \end{subsection}
   
   \begin{subsection}{Überlappende periodische Ablaufplanung}
    \f{Satz:} gegeben sei ein iterativer Problemgraph $G(V,E,s)$. Für jede Kante $e\in E$ sei $w(q(e))$ die Berechnungszeit ihres Quellknotens. Dann ist die Periode nach unten beschränkt durch
     \[ P_{min} = \max \left\{ \left\lceil \frac{\sum_{e\in Z} w(q(e))}{\sum_{e\in Z} s(e)} \right\rceil \bigg | Z\text{ ist Zyklus in }G \right\} \]
     Die Periodenschranke kann in $O(|V|\cdot|E|\cdot \log \sum_{v\in V} w(v))$ berechnet werden.
     
     Ein Problemgraph hat \f{perfekte Rate} wenn es für jeden einfachen Zyklus $Z$ die Summe $\sum_{e\in Z} s(e) = 1$ ist. Hat ein Graph perfekte Rate, so gibt es zu einem Ablaufplan minimaler Periode auch eine statische Bindung.
     
     Da eine zyklostatische Bindung eine Periodizität $K$ hat, kann man sie im Grunde gleichsetzen mit einer statischen Bindung bei einem $K$-fach abgerollten oder entfalteten Problemgraphen (zu jedem Knoten gibt es nun $K$-viele Instanzen usw.).
   \end{subsection}
   
   \begin{subsection}{Periodische Ablaufplanung unter Ressourcenbeschränkungen}
    Diese können wieder mit ILPs gelöst werden.
    
    \f{Satz -- Processor Bound:} Gegeben sei ein iterativer Problemgraph $G = (V,E,s)$ und ein einziger Ressourcetyp (Prozessor) auf dem jeder Knoten $v_i$ in Zeit $w(v_i)$ ausgeführt werden kann. Ferner sei eine Periode $P$ für den Ablaufplan gegeben. Dann gilt für die minimale Zahl von benötigten Ressourcen $\alpha_{min}$:
    \[ \alpha_{min} = \left\lceil \frac{\sum_i w(v_i)}{P} \right\rceil \]
    Das zugehörige ILP in Worten:
    \begin{itemize}
     \item eine binäre Variable $x_{i,t}$ drückt den Ablaufplan aus. Sie ist genau dann $1$, wenn die Aufgabe $v_i$ zum Zeitpunkt $t+nP$ gestartet wird, also $\uptau (v_i) = t+nP$.
     \item die Aufgabe $v_i$ darf nur genau ein mal pro Periode geplant werden.
     \item Es gilt $ \sum_{t=l_i}^{h_i} t\cdot x_{i,t} = \uptau(v_i) $ und somit sagt die Bedingung aus, dass Aufgabe $v_j$ frühestens $w_i - s_{i,j} P$ Zeitschritte später als Aufgabe $v_i$ geplant werden darf, wenn es eine Datenabhängigkeit mit Indexverschiebung $s_{i,j}$ zwischen Aufgabe $v_i$ und Aufgabe $v_j$ gibt.
     \item Ressourcenbeschränkung: Man überlege sich, dass die Ressource $\beta(v_i)$ zum Zeitpunkt $t + nP$ durch $v_i$ nur belegt sein kann, wenn
     \[ \uptau(v_i) \leq t \leq \uptau(v_i) + w_i -1 \text{ oder } \uptau(v_i)-P \leq t \leq \uptau(v_i) + w_i -1 -P \]
    \end{itemize}
   \end{subsection}
   
   \f{Gründe für dynamische Ablaufplanung}
   \begin{itemize}
    \item unbekannte oder datenabhängig stark variierende Länge der Bearbeitungszeiten
    \item Zahl/Art der Jobs vorher nicht bekannt
    \item Datenabhängigkeiten teilweise unbekannt oder variabel
   \end{itemize}
   
   \begin{itemize}
    \item $t_r(v)$ -- Releasezeit, der frühest möglichen Startzeitpunkt eines Jobs. 
    \item $t_d(v)$ -- Deadline, der spätest mögliche Endzeitpunkt eines Jobs.
   \end{itemize}
   
   Mit \f{Dispatchlatenz} $L_D$ bezeichnet man die Zeitspanne, die benötigt wird, um nach Stoppen eines Jobs den nächsten zu starten. 
   
   Die \f{Ressourcenauslastung} $U$ für eine CPU und einen Ablaufplan mit Latenz $L$ ist gegeben durch
   \[ U = \frac{\sum_{v\in V} w(v)}{L} \cdot 100 \]
   Die \f{Flusszeit} einer Aufgabe $v$ ist gegeben durch: $t_F(v) = \uptau_e(v)-t_r(v)$
   
   Die \f{Wartezeit} einer Aufgabe $v$ ist gegeben durch: $t_W(v) = t_F(v)-w(v)$, also die Zeit vom frühest möglichen Startzeitpunkt bis zu ihrem effektiven Ende.
   
   Die \f{Lateness} eines Jobs $v$ ist gegeben durch: $t_L(v) = \uptau_e(v) - t_d(v)$. Sie kann sowohl positiv als auch negativ sein. 
   
   Die \f{Tardiness} eines Jobs $v$ ist gegeben durch: $t_T(v) = \max\{ t_L(v),0 \}$. Diese verwendet man, wenn man nur an echten Zeitverletzungen intressiert ist.
   
   Man kennt \f{time sharing Systeme} vorallem bei Mehrbenutzersystemen (Betriebssysteme) -- man möchte die Ressourcenauslastung maximieren und die Fluss- bzw. Wartezeiten minimieren.
   
   Bei \f{real time operating systems} herrschen harte Zeitbedingungen, daher sind die primären Ziele: 
   \begin{itemize}
    \item minimieren der Lateness, 
    \item minimieren der mittleren Tardiness, 
    \item minimieren der Anzahl von Aufgaben mit Tardiness $> 0$ 
   \end{itemize}
  
  \begin{subsection}{nichtpräemptive dynamische Systeme}
   also von Systemen, die ohne das Unterbrechen von Aufgaben auskommen.
   
   Bei \f{first come first serve (FCFS)} handelt es sich um eine Heuristik, die die Aufgaben in der Reihenfolge einplant, in der sie auftreten. Dabei hängt die mittlere Wartezeit stark von der tatsächlichen Reihenfolge ab.
   
   Bei \f{shortest job first (SJF)} handelt es sich um eine Heuristik, die die Reihenfolge nach der gewichteten Bearbeitungszeit bildet. $w(v)$ Bearbeitungszeit, $c(v)$ Gewicht $\Rightarrow$ Reihenfolge aus $\frac{w(v)}{c(v)}$. Ohne Datenabhängigkeiten liefert dieses Verfahren eine minimale mittlere Wartezeit (Satz von Smith). Dabei bleibt die Priorität der Aufgaben statisch, ändert sich also nicht während der Berechnung. Für Releasezeiten $t_r(v) \neq 0$ oder mit Datenabhängigkeiten allerdings dieses Problem NP-hart.

   Bei \f{earliest deadline first (EDF)} handelt es sich um eine Heuristik, die die Jobs anhand ihrer Deadlines einplant. Sie minimiert die maximale Lateness, wenn alle Jobs Releasezeit $t_r(v)=0$ haben (Jackson Regel). Diese hat ebenfalls statische Prioritäten. Auch bei Releasezeiten $t_r(v) \neq 0$ oder mit Datenabhängigkeiten wird dieses Problem NP-hart.
  \end{subsection}
  
  \begin{subsection}{Präemptive dynamische Planung}
   Alle Jobs können nun in ihrer Bearbeitung unterbrochen werden. 
   
   Bei \f{weighted round robin (WRR)} handelt es sich um eine Heuristik, die die Jobs in eine FIFO Warteschlange einreiht. Jeder Job dieser Schlange wird $c(v)\cdot Q$ Zeiteinheiten lang bearbeitet und dann pausiert und es kommt der nächste Job der Schlange dran. Dabei kommt es nicht zum Aushungern von Jobs, alle sind mal dran. WRR ist einfach zu implementieren, liefert jedoch eine hohe mittlere Wartezeit und kann nicht für Echtzeitbedingungen garantieren.
   
   Auch \f{EDF} lässt sich gut auf präemptive Systeme anwenden. Wenn ein neuer Job bereit wird überprüft man dessen Priorität und unterbricht dann evtl. den aktuell laufenden Job, falls dieser eine niedrigere Priorität aufweist. Falls ein Ablaufplan ohne Datenabhängigkeiten existiert, der alle Deadlines einhält, so existiert auch ein Plan nach EDF, der alle Deadlines einhält (stärker als Jackson, da keine Releasezeiten gefordert).
   
   Die Vorhersagbarkeit des Verhaltens ist ein Problem der präemptiven Planungssysteme, weswegen sie kaum in Echtzeitsystemen mit sicherheitskritischen Deadlines eingesetzt werden. Es ist nicht einmal möglich, stets best- und worst-case Verhalten solcher System anzugeben.
  \end{subsection}
   
  \begin{subsection}{Präemptive periodische Planung}
   Für dynamische Planungsverfahren, bei denen alle Aufgaben unterbrochen werden können.
   
   \f{Periodische Tasks}
   Es wird ein System periodisch auftretender Jobs mit relativen Deadlines $t_d^*(v)$ und relativen Releasezeiten $t_r^*(v)$ definiert, die mit folgenden Releasezeiten und Deadlines assoziiert werden:
   \begin{itemize}
    \item $t_r(v,n) = t_r^*(v)+n\cdot P(v)$
    \item $t_d(v,n) = t_r^*(v)+t_d^*(v)+n\cdot P(v)$
   \end{itemize}
   
   Die \f{ratenmonotone Planung (RM-scheduling)} ist ein solches Verfahren. Es ist prioritätsbasiert und ordnet Tasks ihrer Rate $\frac{1}{P(v)}$.
   
   \pic{RMschedule}


  \end{subsection}
  \end{section}
\end{chapter}
